\section{A possible approach for a social network analysis}
\label{sec:network_analysis}

The following approach to analyze the mice population's social network is based on a selection of methods from the book \textit{Exploring Animal Social Networks}\citep{croft:07}. 

Unfortunately, genetic data to determine kinship in the mice population is not available yet. This circumstance considerably diminishes the validity of our results as relatedness is thought to be a major factor in the formation of social bonds. However, the sex of most mice is known. Therefore the following approach will focus on the differences between the same- and opposite-sex relationships.

As outlined in section \ref{subsubsec:export_options}, the available formats to export the network data using the \textit{miceminer} application, allow to use different software packages to carry out a network analyses. I decided to use the \textit{statnet}\citep{statnet:03} and \textit{igraph}\citep{igraph:06} packages for \lstinline|R|\citep{r:05}. Mostly because \lstinline|R| offers a scripting language which simplifies the execution of repetitive tasks. 

% Started here because chapter intro is not good  -Thomas Netter 21/09/2009 13:49 
% Ok - I changed the beginning of the introduction. But the follwing two paragraphs are relevant and not that bad in my opnion. - rico 

\subsection{Data selection \& edge filtering}
\label{subsec:data_selection}

% do bulleted list and clarify/reformulate sentence below, I don't underst 3 hours & 20 hours -Thomas Netter 21/09/2009 14:05
% This should be clear when one reads the whole thesis or follows the reference to the section - rico 

The network data has been retrieved with minimal time thresholds set to 3 hours and 30 hours per month and range from June 2008 to June 2009. For details about the time threshold see section \ref{subsec:graph_config} on page \pageref{subsec:graph_config}.

Since data is exported per month, we obtain data for 26 networks (13 for each limit).

% bullet bullet bullet!!!! -Thomas Netter 21/09/2009 14:07 
% I hate bullet lists, looks like a PowerPoint presentation ;-) But you told me to start a bullet list with 3 items ore more  - rico
Thresholds are selected based on the following considerations.
\begin{mylist}
\item Whatever the social relation, two mice must spend at least 5 minutes per day together \\ ($\frac{5\:min}{day} * 30\:days\:=\:150\:min\:=\:2.5\:hours \approx 3\:hours$),
\item To build a biologically plausible relationship, two mice should spend at least 1 hour per day together \\ ($\frac{1\:hour}{day} * 30\:days\:=\:30\:hours$).
\end{mylist}

Furthermore, the selection of the 30 hours threshold is justified by the histogram data of monthly meeting duration, for the network data of January 2009, shown in figure~\ref{fig:meeting_frequency_january}.
%<-- january! -Thomas Netter 21/09/2009 14:11 
% Now, that was a detail you freakin perfectionist, not even visible to the reader. But in general, are months in capital letters or not? - rico

The distribution shows, that the threshold of 30 hours is close to the mean and median value. The process of determining the right threshold is called \textit{edge filtering}.

\begin{figure}[htpb]
\begin{center}
  \includegraphics[width=.75\textwidth]{assets/pdf/meeting_frequency_january.pdf}
  \caption[Histogram of monthly meeting duration]{Meeting duration distribution for the network data for January 2009. Additionally, the median and mean values (hh:mm:ss) are indicated.}
  \label{fig:meeting_frequency_january}
\end{center}
\end{figure}   

\subsection{Visual exploration}
\label{subsec:visual_exploration}

The network visualizations in this section have been created using the \textit{network}\cite{network:08} package, which is part of the \textit{statnet}\cite{statnet:03} library. The layout is determined by the \textit{Fruchterman \& Reingold}\cite{fruchterman:91} algorithm, which is a force-directed layouter similar to the one used in the \textit{miceminer} implementation (see section~\ref{subsec:graph_explore}).

Other than the visualization in \textit{miceminer}, the \textit{network} package allows to visualize all network components simultaneously. Moreover, the package is flexible to display selected data or highlight specific nodes in the visualization. 

\subsubsection{General network structure}
\label{subsubsec:vis_general}    

Pictured in figures~\ref{fig:graph_january_3h} and~\ref{fig:graph_january_30h} are networks for January 2009 with the edge filter threshold set to 3 hours and 30 hours, respectively. Although the edges within network components are not visible, the amount of \textit{isolates} - nodes with a degree of zero - is noticeably higher in the network with an edge filter value set to 30 hours. \textit{Isolates} arise from the filter process. Edges with a value lower than the threshold are removed from the network. If all the edges for a certain node are below threshold, this node becomes an \textit{isolate}. 

\begin{figure}[htpb]% 
	\centering 
	\subfloat[Edge filter threshold set to 3 hours][Edge filter threshold set to 3 hours.]{
					\label{fig:graph_january_3h} %
					
					\includegraphics[width=.6\textwidth]{assets/pdf/graph_january_3h.pdf}
				}% 
	\qquad 
	\subfloat[Edge filter threshold set to 30 hours][Edge filter threshold set to 30 hours.]{
					\label{fig:graph_january_30h}%
					\includegraphics[width=.6\textwidth]{assets/pdf/graph_january_30h.pdf}
				} 
	\caption[Network visualizations with different edge filter thresholds]{A visualization of the network for January 2009 with the edge filter threshold set to 3 hours \subref{fig:graph_january_3h} and 30 hours \subref{fig:graph_january_30h}. Isolates are highlighted in red.}  
	 
\end{figure}   

\clearpage

To draw some clearer image of the network and since we are interested in the differences of the networks of same- and opposite-sex, the nodes in figures~\ref{fig:graph_january_3h_gender}~and~\ref{fig:graph_january_30h_gender} are colored based on their sex.

Figure \ref{fig:graph_january_30h_gender} shows that females tend to be located more in the center of the components, whereas male mice are found in the periphery. Furthermore, the same figure reveals that the number of \textit{isolates} is greater for males than for females. This can be elaborated by visualizing the same-sex (Figs.~\ref{fig:graph_january_30h_ff} and \ref{fig:graph_january_30h_mm} ) and opposite-sex networks (Fig.~\ref{fig:graph_january_30h_fm}). The visual comparison clearly unveils the higher connectivity within the female mice.

\begin{figure}[htpb]% 
	\centering 
	\subfloat[Network with edge filter threshold set to 3 hours and node coloring based on the  sex of the mouse][Edge filter threshold set to 3 hours.]{
					\label{fig:graph_january_3h_gender} %
					\includegraphics[width=.45\textwidth]{assets/pdf/graph_january_3h_gender.pdf}
				}% 
	\qquad 
	\subfloat[Network with edge filter threshold set to 30 hours and node coloring based on the sex of the mouse][Edge filter threshold set to 30 hours.]{
					\label{fig:graph_january_30h_gender}%
					\includegraphics[width=.45\textwidth]{assets/pdf/graph_january_30h_gender.pdf}
				} 
			
	\caption[Network visualizations with different edge filter values]{A visualization of the network for January 2009 with the edge filter threshold set to 3 hours \subref{fig:graph_january_3h} and 30 hours \subref{fig:graph_january_30h}. Additionally, the nodes are colored according to the sex of the mice. Female mice are colored pink, males light blue, and mice with an unknown sex grey.}  
\end{figure}

% Why suddenly Fig. and Figs. for references to figures? - rico

Even in the network with the edge filter set to 30 hours, the entanglement within network components does not allow identification of all individual edges. However, the network filtered with the higher threshold draws a much clearer image of the network structure. For that reason, the following analysis is carried out using networks with an edge filter set to 30 hours.


\begin{figure}[htpb]% 
	\centering 
	\subfloat[Network of female mice][Network of female mice]{
					\label{fig:graph_january_30h_ff} %
					\includegraphics[width=.45\textwidth]{assets/pdf/graph_january_30h_ff.pdf}
				}
	\qquad 
	\subfloat[Network of male mice][Network of male mice]{
					\label{fig:graph_january_30h_mm}%
					\includegraphics[width=.45\textwidth]{assets/pdf/graph_january_30h_mm.pdf}
				}
	\qquad  
	\subfloat[Inter-sex network][Inter-sex network]{
					\label{fig:graph_january_30h_fm}%
					\includegraphics[width=.45\textwidth]{assets/pdf/graph_january_30h_fm.pdf}
				} 
				
	\caption[Network visualizations of the same- and opposite-sex networks]{Visualizations of the same- and opposite-sex networks for January 2009 with the edge filter value set to 30 hours.}
	\label{fig:inner_inter_gender} 
	 
\end{figure}

\subsubsection{Identifying individuals}
\label{subsubsec:vis_individuals}    

After visually examining the general network structure, the visualizations in figure~\ref{fig:graph_january_30h_node_based_measures} are adapted to emphasize the role of individual network nodes, based on the node based measures introduced in section \ref{subsec:node_based}.  

\begin{figure}[htpb]% 
	\centering 
				
	\subfloat[Network visualization where node size is proportional to node average path length][Network visualization where node size is proportional to node average path length (larger node size means lower value).]{

					\label{fig:graph_january_30h_apl}%
					\includegraphics[width=.45\textwidth]{assets/pdf/graph_january_30h_apl.pdf}
				}
	\qquad 
	\subfloat[Network visualization where node size is proportional to node clustering coefficient][Network visualization where node size is proportional to node clustering coefficient (larger node size means higher value).]{
					\label{fig:graph_january_30h_cc}%
					\includegraphics[width=.45\textwidth]{assets/pdf/graph_january_30h_cc.pdf}
				}
	\qquad 			
	\subfloat[Network visualization where node size is proportional to node degree][Network visualization where node size is proportional to node degree (larger node size means higher value).]{
					\label{fig:graph_january_30h_degree}
					\includegraphics[width=.45\textwidth]{assets/pdf/graph_january_30h_degree.pdf}
				}
	\qquad 
	\subfloat[Network visualization where the node size is proportional to node betweenness][Network visualization where the node size is proportional to node betweenness (larger node size means higher value).]{
					\label{fig:graph_january_30h_betweenness}
					\includegraphics[width=.45\textwidth]{assets/pdf/graph_january_30h_betweenness.pdf}
				} 		 				
		
	\caption[Network visualizations where node size is proportional to node based measures]{Network visualizations for January 2009, where node size is proportional to a node based measure.}
	\label{fig:graph_january_30h_node_based_measures} 
	 
\end{figure}

In all but the visualization of betweenness (Fig.~\ref{fig:graph_january_30h_betweenness}), no nodes stand out. This is coherent with the observation that network components are strongly connected. Consequently, the clustering coefficient is generally high and average path lengths are low.  

% I stop here for today -Thomas Netter 21/09/2009 14:43
% Ok, try to remove the obsolete the's from here - rico

A closer look at figure~\ref{fig:graph_january_30h_betweenness} reveals that most of the nodes with a high betweenness value are \textit{Cut-Points} as well. A \textit{Cut-Point} is a node whose deletion increases the number of components in the network\citep{pajek:03}. These nodes are potentially interesting, because they act as a bottleneck when information has to be passed from one sub-network to the other. In this case, however, all \textit{Cut-Points} connect at most two other nodes to a component (Fig.~\ref{fig:graph_january_3h_cutpoints}), which reduces their general importance. Hence, we focus on nodes with a high betweenness value, that are not \textit{Cut-Points} of the kind just explained. The resulting visualization (Fig.~\ref{fig:graph_january_3h_cutpoints_betweenness}) unveils two female mice with a potentially important position in the network. 

\begin{figure}[hpbt]% 
	\centering 
	\subfloat[Network visualization with highlighted \textit{Cut-Points}][Network visualizations for January 2009 with highlighted \textit{Cut-Points}.] {
		\label{fig:graph_january_3h_cutpoints}
  		\includegraphics[width=.6\textwidth]{assets/pdf/graph_january_30h_cutpoints.pdf}
	}
	\qquad
	\subfloat[Network visualization with nodes highlighted, that have a high betweenness value and are not \textit{Cut-Points}][The larger nodes have a high betweenness value but are not \textit{Cut-Points}.] {
 		\label{fig:graph_january_3h_cutpoints_betweenness}
 		\includegraphics[width=.6\textwidth]{assets/pdf/graph_january_30h_cp_bet.pdf}
  	}
  	
  	\caption[Network visualization of \textit{Cut-Points} and such with a high betweenness value which are not \textit{Cut-Points}]{Identifying \textit{Cut-Points} \subref{fig:graph_january_3h_cutpoints} and such nodes with a high betweenness value which are not \textit{Cut-Points}~\subref{fig:graph_january_3h_cutpoints_betweenness}.}
  	
\end{figure}

\clearpage

\subsection{Quantitative analysis}
\label{subsec:quantitative_analysis}

Aside of quantifying the node based measures, which has already been done for the visualizations, studying the distribution and mix of these measures provides insights into the mechanisms that led to an observed network structure.  

\subsubsection{General network structure}
\label{subsubsec:general_structure}

A basic question, if we consider the network depicted in figure \ref{fig:graph_january_30h_gender}, is how separation of the network in several components evolved. To explore this topic, it was determined in which nestboxes the meetings occurred in January 2009. These findings can be mapped to the components found in the network. The results are shown in table \ref{tab:comp_box_meet_dist}.

\begin{table}
\begin{center}
\small
\renewcommand\arraystretch{1.1}
\begin{tabularx}{\textwidth}{^X^X^X}
\hline
\textbf{Component size} &	\textbf{Nestboxes in which meetings occured}	&	\textbf{Sum of meeting durations (h:m:s)} \\\hline
18	& \textcolor{red}{15}	& \textcolor{red}{2006:35:05} \\
 	& \textcolor{red}{18}	& \textcolor{red}{1800:24:34} \\
	& \textcolor{red}{13}	& \textcolor{red}{1630:43:24} \\
	& \textcolor{red}{12}	& \textcolor{red}{1507:04:51} \\
	& 11	& 36:26:24 \\
	& 16	& 00:21:05 \\
	& 17	& 00:16:42 \\
	& 09	& 00:01:31 \\\hline

18	& \textcolor{red}{27}	& \textcolor{red}{2735:14:47} \\
	& \textcolor{red}{21}	& \textcolor{red}{950:02:00} \\
	& \textcolor{red}{29}	& \textcolor{red}{713:50:05} \\
	& 25	& 187:48:39 \\
	& 28	& 141:33:31 \\
	& 22	& 134:26:27 \\
	& 23 	& 12:36:56 \\
	& 24	& 11:02:57 \\\hline

17	& \textcolor{red}{36}	& \textcolor{red}{5676:16:38} \\
	& \textcolor{red}{37}	& \textcolor{red}{1716:46:11} \\
	& 35	& 254:34:27 \\
	& 34	& 228:01:20 \\
	& 40	& 179:18:27 \\
	& 38	& 23:19:02 \\
	& 09	& 00:00:12 \\
	& 39	& 00:00:8 \\\hline

13	& \textcolor{red}{09}	& \textcolor{red}{2339:51:41} \\
	& \textcolor{red}{08} 	& \textcolor{red}{1984:33:20} \\
	& 10	& 232:50:42 \\
	& 05	& 11:01:31 \\
	& 03 	& 02:05:59 \\
	& 11 	& 00:57:41 \\
	& 06	& 00:12:20 \\
	& 07	& 00:02:50 \\\hline
	
10	& \textcolor{red}{02}	& \textcolor{red}{2667:49:11} \\
	& \textcolor{red}{01}	& \textcolor{red}{2075:11:12} \\
	& 04	& 01:26:32 \\
	& 03	& 00:00:32 \\
	& 12	& 00:00:32 \\
	& 09	& 00:00:05 \\\hline

2	& \textcolor{red}{19}	& \textcolor{red}{52:41:08} \\
	& 20	& 05:08:17 \\
	& 18	& 00:00:30 \\\hline

\end{tabularx}
\captionof{table}{Allocation of nestboxes, in which meetings occurred in January 2009, to the components observed in the network. Additionally the size of components (number of mice making up the component) and the sum of meeting durations for the boxes is included (according to the \lstinline|<maxStay>| value explained in section \ref{subsec:miceminer_config}, meetings with a duration of stay longer than 9 hours have been excluded from the sum).}
\label{tab:comp_box_meet_dist}
\end{center}
\end{table} 

Interestingly, if we compare this allocation to the placement of nestboxes and dividers in the barn (see figure~\ref{fig:barn_schema} on page \ref{fig:barn_schema}), from the six components, five are restricted to one of the four segments (A, B, C, D). Furthermore, the number of boxes where most of the meetings of a component occur (values colored in red in table \ref{tab:comp_box_meet_dist}), is between 2 and 4. 
 
\subsubsection{Separation of node based measures by sex}
\label{subsubsec:nbm_dist}

Plotted in figure~\ref{fig:node_based:measures_dist} are the frequency distributions of the node based measures for female and male mice. An analysis of the segregation by sex contributes to understanding the underlying mechanisms that form the network.

\begin{figure}[htpb]% 
	\centering 
				
	\subfloat[Average path length]{
					\label{fig:graph_january_30h_apl_dist}%
					\includegraphics[width=.45\textwidth]{assets/pdf/graph_january_30h_apl_dist.pdf}
				}
	\qquad 
	\subfloat[Clustering coefficient]{
					\label{fig:graph_january_30h_cc_dist}%
					\includegraphics[width=.45\textwidth]{assets/pdf/graph_january_30h_cc_dist.pdf}
				}
	\qquad 			
	\subfloat[Degree]{
					\label{fig:graph_january_30h_degree_dist}
					\includegraphics[width=.45\textwidth]{assets/pdf/graph_january_30h_degree_dist.pdf}
				}
	\qquad 
	\subfloat[Betweenness]{
					\label{fig:graph_january_30h_betweenness_dist}
					\includegraphics[width=.45\textwidth]{assets/pdf/graph_january_30h_bet_dist.pdf}
				} 		 				
				
	\caption[Distribution of the node based measures split up by sex.]{Distribution of the node based measures, for the network data of January 2009 with an edge filter value of 30 hours. Values for female mice are colored pink, males are light blue.}
	 \label{fig:node_based:measures_dist}
\end{figure}

Although the charts do not show a clear separation, network visualizations support the observation that females are better connected than males (Figs.~\ref{fig:inner_inter_gender} and \ref{fig:graph_january_30h_node_based_measures}). Figure~\ref{fig:graph_january_30h_degree_dist}, for instance, shows that there are more male \textit{isolates} than female. Furthermore, female mice with a high betweenness value are easy to identify in figure \ref{fig:graph_january_30h_betweenness_dist}.  

The simplest way to quantify the segmentation is to calculate the means by sex (see table \ref{tab:means_nbm}). There seems to be an explicit difference in the betweenness and the degree values.

\begin{table}
\begin{center}
\small
\renewcommand\arraystretch{1.2}
\begin{tabular}{+l^l^l^l}
\hline
\rowstyle{\bfseries}
Measure &	Mean females	&	Mean males	& range of values (min /max) \\ \hline
Average path length	& 1.27	& 1.22	&  1 / 2.94 \\
Clustering coefficient	& 0.70	& 0.59	& 0 / 1 \\
Degree	& 8.2	& 6	& 1 / 14 \\
Betweenness	& 5.43	& 1.54	& 0 / 40 \\ \hline
\end{tabular}
\captionof{table}{Mean values of the node based measures separated by sex. Additionally the range of the values is indicated.}
\label{tab:means_nbm}
\end{center}
\end{table}

However, a better method to determine separation of these values by sex is to use a \textit{Mann-Whitney} test\citep{siegel:88}. For this test, all values for a node based measure are arranged into a single ranked series. 

Let $R_f$ denote the ranks of females in this series, and $n_f$, $n_m$ be the numbers of female and male nodes. The equation to get the coefficient $U_f$ looks like

\begin{equation}
U_f = n_fn_m\frac{n_f(n_f + 1)}{2} - R_f
\label{eq:mann_w}
\end{equation}  

$U_f$ is then scaled by $n_fn_m$

\begin{equation}
u_f = \frac{U_f}{n_fn_m}
\label{eq:mann_w_norm}
\end{equation}
 
The value for $u_f$ is always between 0 and 1. If females occupy the lowest $n_f$ ranks,  $u_f = 0$, and if males occupy the lowest $n_f$ ranks, $u_f=1$. For a perfect mixing of ranks, $u=0.5$.

The results are listed in table \ref{tab:u_test}. For calculation of the average path length and clustering coefficient, \textit{isolates} were not included, because they would bias the result.

\begin{center} 
\renewcommand\arraystretch{1.2}
\begin{tabular}{+l^l^l^l}
\hline
\rowstyle{\bfseries}
Measure &	Females	& Males	& $u$ \\ \hline
Average path length	&	41	&	35	& $0.38$\\
Clustering coefficient	&	41	&	35	&  $0.65$ \\
Degree	&	45 	& 	47 	& $0.62$	\\
Betweenness	&	45	&	47	&	$0.67$ \\
\hline
\end{tabular}
\captionof{table}{$u$ values for node based measures calculated using a \textit{Mann-Whitney} test.}
\label{tab:u_test}
\end{center}

Results also show a tendency for better connectedness of among female mice. Note that lower values for the average path length means a better rank.

To test whether values of $u$ are statistically significant, they should be compared to those calculated for a set of random graphs with a fixed degree distribution\citep{croft:07, newman:02a}. However, no implementation could be found, in the \lstinline|R| packages used in this analysis, to generate such random networks for unconnected graphs (network which consists of several components). 

\subsubsection{Assortivity coefficient}
\label{subsubsec:assortivity}     

The assortivity coefficient\citep{newman:03} measures association patterns in social networks. If $e_{ij}$ is the fraction of edges in a network that connect nodes of type $i$ to nodes of type $j$, the assortivity coefficient is defined as

\begin{equation}
r = \frac{ \sum_i e_{ii} - \sum_{ijk} e_{ik} e_{jk} }{ 1 - \sum_{ijk} e_{ik} e_{jk}}
\label{eq:ass_coeff}
\end{equation} 

where $k$ is a dummy variable used to iterate the sum\citep{lusseau:04}. This quantity equals $1$ when we have perfect assortative mixing, meaning that all nodes of a category are connected to nodes of the same category. For a disassortative mixing, when every node of a type is connected to a node of another category, the value of $r$ lies between $-1 \leq r \leq 0$.  

\paragraph{Correlation by sex}
\label{para:gender_corr}

In our data, one category we can test for is the sex of the mice. Table \ref{tab:mm} shows the \textit{mixing matrix}, which discloses the quantities of same- and opposite-sex connections in the network of January 2009 (e.g. there are 154 edges between males and females).

%rico

\begin{center}
\renewcommand\arraystretch{1.2}
\newcolumntype{H}{>{\centering\bf}p{0.3cm}}
\begin{tabular}{+H|^c^c^c}
\rowstyle{\bfseries}
	&	f	&	m	&	u \\\hline
f	&	103	&	154	&	10 \\
m	&	154	&	60	&	8 \\
u	&	10	&	8	&	0 \\	
\end{tabular}
\captionof{table}{\textit{Mixing matrix} of the sex category for the network data of January 2009.}
\label{tab:mm}
\end{center}

This distribution yields an $r$ of $0.010$ which implies a disassortative mixing. Edges that contain nodes of unknown sex were not included in the calculation. 

For a mixing matrix with several categories this value would be tested for statistical significance, for example using the \textit{jackknife}\citep{newman:03} method,  

\begin{equation}
\sigma_r^2 = \sum_{i=1}^M(r_i -r)^2
\label{eq:ass_coeff_gender}
\end{equation}  

where $r_i$ is the $r$ value of the network in which the $i$-th edge is removed, to determine the standard deviation, followed by a simple $t$-test\citep{snijders:99}.

However, this is not applicable for the present data, because $r$ values calculated according to the \textit{jackknife} method are not subject to Gaussian distribution. Each iteration in the \textit{jackknife} method either removes a same- or opposite-sex edge. Therefore, only three different $r$ values occur.

\paragraph{Degree correlation}
\label{para:degree_corr}
 
Another form of assortative mixing is \textit{degree correlation}, which measures assortative mixing by degree. The question behind this value is whether individuals with high degrees tend to be connected to others with a high degree\citep{croft:07}. Degree correlation for the network of January 2009 using the \textit{Pearson correlation}, as suggested by \textit{Newman}\citep{newman:02}, results in $r_p = 0.373$. \citep{newman:03a}.  

There is no general interpretation of the $r_p$ value. However, we can compare it with the values calculated for other networks. Interestingly, many social networks exhibit a positive degree correlation, whereas metabolic networks, food webs, and neural networks usually have a negative degree correlation\citep{newman:03a}. \textit{Lusseau et al.}\citep{lusseau:06} for example, reported a value of $r_p = 0.170$ for a social network of bottlenose dolphins, and \textit{Croft et al.}\citep{croft:05} found degree correlation values between $0.28$ and $0.7$ for 5 populations of small freshwater fish.      
 
\subsection{Longitudinal exploration of the network data}
\label{subsec:longitudinal}

After the visual exploration and the examination of some quantitative network measures on the basis of a single network, we expand these methods to a series of networks. The series includes data from June 2008 to June 2009. Although these 13 networks do not reveal annual periodicities, as such a study would need to be carried out with network data for several years, tendencies can nevertheless be spotted.
 
To identify these tendencies, rather than analyzing the data based on known methods to study the longitudinal dynamics in networks\citep{snijders:05}, most of the already introduced quantitative measures have been computed for the whole set.

\subsubsection{Nodes}

Pictured in figure~\ref{fig:long_node}, is the quantity of mice for each of the thirteen months. These numbers only include mice for which meetings have been recorded. The blue values indicate the entire set of mice per month. Values in pink, light blue, and grey, represent the quantity of female, male, and mice with an unknown sex, respectively. Unless otherwise stipulated, the color code is the same for all figures in this section.

Notice the convergence of quantities for female and male mice from September 2008 to April 2009. This allows for a more significant comparison of these networks, since many methods and measures are in some way dependent on the number of nodes.    

\begin{figure}[htpb]
\begin{center}
  \includegraphics[width=.6\textwidth]{assets/pdf/long_nodes.pdf}
  \caption[Number of mice over the months]{Quantity of mice for each month. The solid blue line indicates the sum of female (pink), male (light blue) and mice with an unknown sex (grey).}
  \label{fig:long_node}
\end{center}
\end{figure}

\clearpage

\subsubsection{Edges}

Figure \ref{fig:long_edges} shows the number of edges for the networks. Furthermore, the number of edges between females (pink), males (light blue) and between males and females (grey) are indicated. The sum of edges (colored blue) includes edges with mice of unknown sex. However, this fraction is not indicated in the figure.

\begin{figure}[htpb]
\begin{center}
  \includegraphics[width=.6\textwidth]{assets/pdf/long_edges.pdf}
  \caption[Number of edges over the months]{Number of edges over the months.}
  \label{fig:long_edges}
\end{center}
\end{figure}

The chart suggests a seasonal trend for the mice connectedness during summer and winter. Even though the number of mice is lower in summer than in winter (Fig.~\ref{fig:long_node}), the differences in mice connectedness is clearly visible when the networks are compared visually. Figure \ref{fig:june_jan_side} shows networks for June 2008 and January 2009 side by side. Clearly, the density of edges in the network for January is much higher than for June. The density of a network can be quantified by calculating the fraction of possible edges in the network, given by:  

\begin{equation}
\rho = \frac{E}{E_max} = \frac{2E}{n(n-1)}
\label{eq:density}
\end{equation}      

where $E$ denotes existing edges, and $n$ is the number of nodes. The densities for June 2008 and January 2009 are $\rho_{jun.} = 0.01$ and $\rho_{jan} = 0.14$. The factor of 14 between these two values is even underestimated, since the number of possible edges for June is $2701$\footnote{$E_{max} = (\frac{1}{2})n(n-1) = (\frac{1}{2}) 74(74 -1) = 2701$} whereas for January 2009 it is 4656\footnote{$E_{max} = (\frac{1}{2}) 97(97-1)  = 4656$}. Hence, the mice are much better connected in winter than in summer.  

\begin{figure}[htpb]% 
	\centering 
	
	\subfloat[June 2008]{
					\label{fig:graph_jun_30h}%
					\includegraphics[width=.45\textwidth]{assets/pdf/graph_june_08_30h.pdf}
				}	
	\qquad 			
	\subfloat[January 2009]{
					\label{fig:graph_jan_30h}%
					\includegraphics[width=.45\textwidth]{assets/pdf/graph_january_30h_gender.pdf}
				}
	\caption[Network visualizations for June 2008 and January 2009]{Network visualizations for June 2008 and January 2009.}
	 \label{fig:june_jan_side}
\end{figure} 

The number of edges between members of opposite sex is usually higher than between members of the same sex. Moreover, female mice seem better connected among each other than males. The most significant differences are found for the networks of January and February. The number of mice in these two months is almost equal for both sexes (Fig.~\ref{fig:long_node}), but the number of edges running between females is noticeably higher. 
 
      
\subsubsection{Components}

The number of network components (Fig.~\ref{fig:long_comps}) reveal how fragmented the network is. \textit{Isolates} have not been counted as components in this calculation. Values for female and male mice indicate in how many network components one or more individual of a given sex is found. Therefore, an overlap of values for the sexes and the number of components (colored blue) indicates a good mix of sexes within components. This is for example the case from October to December. Consequently, we may expect disassortative mixing by sex (see section \ref{para:gender_corr}) for these months. The dynamics of the correlation coefficient by sex is shown in figure~\ref{fig:long_gender_corr}. Indeed, over that period the coefficient $r$ is around 0, which speaks for disassortative mixing.

\begin{figure}[htpb]
\begin{center}
  \includegraphics[width=.6\textwidth]{assets/pdf/long_comps.pdf}
  \caption[Number of components over the months]{Number of components over the months. \textit{Isolates} are not included in the calculation.}
  \label{fig:long_comps}
\end{center}
\end{figure}


% \subsubsection*{Clustering coefficient}
% 
% \begin{figure}[htpb]
% \begin{center}
%   \includegraphics[width=.6\textwidth]{assets/pdf/long_cc.pdf}
%   \caption[clustering coefficient]{clustering coefficient}
%   \label{fig:long_cc}
% \end{center}
% \end{figure} 

\subsubsection{Degree}

Figure~\ref{fig:long_degree} shows the nodes mean degree over the months. As already assumed based on observations made in the single network in section \ref{subsubsec:nbm_dist}, female mice retain a higher mean degree over the whole period. This is another argument that supports the fact that female mice are socially better or more widely connected.


\begin{figure}[htpb]
\begin{center}
  \includegraphics[width=.6\textwidth]{assets/pdf/long_degree.pdf}
  \caption[Mean degree of the nodes over the months]{Mean degree of the nodes over the months.}
  \label{fig:long_degree}
\end{center}
\end{figure} 


\subsubsection{Betweenness}
\label{subsubsec:long_betweenness}

As already mentioned in the section where the betweenness value has been introduced (see section \ref{subsubsec:node_between} on page \pageref{subsubsec:node_between}), the betweenness usually correlates with the degree of a node. The mean betweenness values for the monthly networks shown in figure~\ref{fig:long_betweenness}, mostly support this rule. However, the separation of the betweenness values for the sexes is more distinct compared to the degree values (Fig.~\ref{fig:long_degree}). An exception to the rule are the values found in March.   

\begin{figure}[htpb]
\begin{center}
  \includegraphics[width=.6\textwidth]{assets/pdf/long_betweenness.pdf}
  \caption[Mean betweenness of the nodes over the months]{Mean betweenness of the nodes over the months.}
  \label{fig:long_betweenness}
\end{center}
\end{figure}

In March, males show a higher mean betweenness than females. Because this observation does not correlate with the mean degree values, we expect some males to be \textit{brokers} according to the definition in section \ref{subsubsec:node_between}. Network visualizations shown in in figure~\ref{fig:graphs_march} identify two male mice with an exceptionally high betweenness value (figure \ref{fig:graph_mar_prop_bet}) that are  \textit{Cut-Points} (figure \ref{fig:graph_mar_hl_cp}) as well. Such nodes, and especially the edges between them, may be of great interest to analyze single edges in the network.   

\begin{figure}[htpb]% 
	\centering 
	
	\subfloat[Node size proportional to betweenness value]{
					\label{fig:graph_mar_prop_bet}%
					\includegraphics[width=.45\textwidth]{assets/pdf/graph_march_09_30h_prop_bet.pdf}
				}	
	\qquad 			
	\subfloat[Highlighted \textit{Cut-Points}]{
					\label{fig:graph_mar_hl_cp}%
					\includegraphics[width=.45\textwidth]{assets/pdf/graph_march_09_30h_hl_cutpoints.pdf}
				}
	\caption[Network visualizations for March 2009]{Network visualizations for March 2009, where node size is proportional to node betweenness \subref{fig:graph_mar_prop_bet}, and \textit{Cut-Points} are highlighted \subref{fig:graph_mar_hl_cp}.}
	 \label{fig:graphs_march}
\end{figure} 


\subsubsection{Correlation by sex}

Figure~\ref{fig:long_gender_corr} shows the assortivity coefficients $r$ for correlation by sex. A clear departure from a disassortative to an assortative mixing is shown. However, $r$ values for sparse networks such as that of June 2009 ($\rho = 0.02$), may not be significant. 

\subsubsection{Degree correlation coefficient}

No periodicity, stability, nor any tendency is visible in the degree correlations shown in figure~\ref{fig:long_degree_cor}. However, values for the coefficient are positive over all months. This is consistent with values determined for other social networks (see section \ref{para:degree_corr}).

\begin{figure}[p]
\begin{center}
  \includegraphics[width=.6\textwidth]{assets/pdf/long_gender_corr.pdf}
  \caption[Assortivity coefficient of sexes over the months]{Assortivity coefficient of sexes over the months.}
  \label{fig:long_gender_corr}
\end{center}
\end{figure}  

\begin{figure}[p]
\begin{center}
  \includegraphics[width=.6\textwidth]{assets/pdf/long_degree_corr.pdf}
  \caption[Degree correlation coefficient]{Monthly degree correlation coefficient.}
  \label{fig:long_degree_cor}
\end{center}
\end{figure} 


\subsection{Identifying potentially interesting individuals}
\label{subsec:follow_individuals}

The idea presented in section \ref{subsubsec:vis_individuals} to identify mice at important positions in the network based on the betweenness values has been applied to all months with a density $\rho \geq 0.05$. The method to detect these nodes is as follows:

\begin{mylist}
\item First, the nodes with the five highest betweenness values in the network are shortlisted,
\item Then each of the nodes in the shortlist is removed from the network,
\item Following that, the number and size of components before and after the deletion is compared,
\item Finally, a node is assessed as a result, if the node removal does not create an additional component with size 2 or less.  
\end{mylist}

Table \ref{tab:foll_bet} lists the resulting nodes found in the networks.

\begin{center}
\small
\renewcommand\arraystretch{1.1}% (MyValue=1.0 is for standard spacing)
\begin{tabular}{+l|^l|^l|^l}
\hline
\rowstyle{\bfseries}
Sept. `08 & Oct. `08 & Nov. `08 & Dec. `09 \\\hline
0006CC79CA	(m) & 00069552DA (f) & 0006B8D0DF (f) & 0006CD3416 (f)  \\
0006B9D3CB	(m) & 0006BA0C9F (f) & 0006B9C5E8 (f) & 0006B8E0D4 (f)  \\
				& 0006B9C5E8 (f) & 0006CC62D7 (f) &	 \\
				& 0006B9D225 (f) & 0006CC62BF (m) &	 \\\hline					
\multicolumn{4}{l}{} \\\cline{1-3}
\rowstyle{\bfseries}
Jan. `09 & Feb. `09 & \multicolumn{2}{l}{\textbf{Mar. `09}} \\\cline{1-3}
0006B9CCE9 (f) 	& 0006B9BAB9 (f) 	& \multicolumn{2}{l}{0006CD4748 (f)} \\
0006B9D225 (f) 	& 0006B8CC8C (f) 	& \multicolumn{2}{l}{0006CC69BE (m)} \\
				& 0006CD3416 (f) 	& \multicolumn{2}{l}{0006CD5439 (m)} \\
				& 0006B8E0D4 (f) 	& \multicolumn{2}{l}{0006954D6A (f)} \\			
				&					& \multicolumn{2}{l}{0006CC5D4D (f)} \\\cline{1-3}					
\end{tabular}
\captionof{table}{Potentially interesting nodes identified by a high betweenness value for the months between October 2008 and March 2009.}
\label{tab:foll_bet}
\end{center}

Due to the observations made for the betweenness value (see section \ref{subsubsec:long_betweenness}), it is not surprising that most of the found nodes are females (14 females and 5 males). A closer look at the table reveals, that some mice that occupy such a position, are found in two different months (see table \ref{tab:foll_bet_more}). By means of the network visualizations for December and February, shown in figure~\ref{fig:pi_nodes_dec_feb}, one can identify the positions of the two RFID's (0006B8E0D4 and 0006CD3416) in the network.

\begin{center}
\small
\renewcommand\arraystretch{1.5}% (MyValue=1.0 is for standard spacing)
\begin{tabular}{+l|^l}
\hline
\rowstyle{\bfseries}
RFID (sex)	&	Months \\\hline
0006B8E0D4 (f)	& December / February \\
0006CD3416 (f)	& December / February \\
0006B9C5E8 (f)	& October / November \\
0006B9D225 (f)	& October / January \\ 
\hline											
\end{tabular}
\captionof{table}{RFID's with high betweenness values that are found in two months.}
\label{tab:foll_bet_more}
\end{center}

\begin{figure}[htpb]% 
	\centering 
	
	\subfloat[December 2008]{
					\label{fig:pi_nodes_dec}%
					\includegraphics[width=.45\textwidth]{assets/pdf/pi_nodes_dec.pdf}
				}	
	\qquad 			
	\subfloat[February 2009]{
					\label{fig:pi_nodes_feb}%
					\includegraphics[width=.45\textwidth]{assets/pdf/pi_nodes_feb.pdf}
				}
	\caption[Network visualizations for December 2008 and February 2009 ]{Network visualizations for December 2008 \subref{fig:pi_nodes_dec} and February 2009 \subref{fig:pi_nodes_feb}. Nodes with a high betweenness value found in both months are highlighted in red.}
	 \label{fig:pi_nodes_dec_feb}
\end{figure} 

\pagebreak

Other approaches can be used to identify potentially interesting nodes. For instance, one could choose a mouse based on observations made in the barn and trace its locations in the network. One may also track mice that are also connected in summer.

\subsection{Discussion}
\label{subsec:discussion}

We showed that meetings which form a network component were recorded at a strictly limited number of nestboxes. These nestboxes are located in a restricted area of the barn. Therefore, we suggest some kind of a nestbox territory. Mice establish social relationships - in the way we measured it - only in a limited number of nestboxes, that are located close to each other. Such territories may even be defended against other individuals or social groups.

This raises questions whether the reproductional success of female mice depends on such a territorial behavior. For instance, do females, that occupy more boxes during winter show a higher reproductive output in the next spring. Or is the size of the group, or even the network component, an indicator of the reproductive success of female mice. However, the group size may depend on other factors such as the available space in a nestbox, as well as thermoregulatory aspects.

Interestingly, few components could be found which consist exclusively of male mice. This indicates that male mice try to dominate a group of female mice. For the female mice in the group, however, we expect formation of social bonds to share brood-rearing duties such as nursing. It is possible that the present network data, paired with genetic information and observations made in the barn, may help identify reproductive groups. Once identified, the social network analysis could contribute to study the dynamics within such a group.     

The observed seasonal variation in mice networking can arise from different reasons. One would be that mice spend more time together in the nestboxes to keep each other warm during winter. Energy costs for thermoregulation may be considerably lower if mice lay close to each other in a covered nestbox laid-out with hew and straw. A second reason could be found in the seasonal discrepancy of competitive behavior for reproduction. During summer, competition among male mice to get access to females is high. Wounded males, found regularly at such time, are a clear sign for this rivalry. Female mice seem to compete among each other to acquire and defend safe nestboxes where they raise their offspring. During winter, however, reproductive activity is low. A hypothesis is that energy costs to simultaneously wean a litter and retain thermoregulation are too high, even when food is provided ad libitum. This hypothesis goes well  with the former reason that thermoregulation costs play an important role. Thus, both hypotheses need not necessarily be understood as alternatives.    

The few edges that exist in the networks during summer, primarily exist between females. We assume these are collaborative females, that raise their pups in a communal nest. This can be verified once genetic data is available.

As already mentioned, female mice tend to be connected to more individuals as males. Furthermore, based on methods used in this approach, they occupy most of the important positions in the network. However, the weight of the edges has been neglected in the analyzed networks. It would be interesting to carry out the analysis using methods which take weights into account. Comparison of results would show, whether females are not only better, but also more strongly connected.

With the measures and methods used in this approach, networks have mainly been analyzed as a whole. Little attention was paid to the individual edges. Note that it would be quite easy to locate a mouse or a potentially interesting edge using the \textit{miceminer} interface.

In summary, it can be said that evaluation of the social network contributes to the understanding of factors that underlie the social structure in this mouse population. Additionally, examination of the network helps raise questions that may not be asked using conventional methods. The social network approach presented here includes only a small set of known methods in a research field that gains more and more attention. With that in mind, there are certainly a lot of research questions that can be tackled using the social network analysis, even in biology.