% ===============================
% Introduction
% ===============================
\newpage

\section{Introduction}
\label{sec:introduction}

In May 2008, the research group around Prof. Barbara K\"onig at the Institute of Zoology of the University of Zurich decided to take their research in the field of Animal Behaviour to the next level. Especially interested in the complex social behaviour of house mice \textit{(Mus domesticus)} (see section \ref{subsec:socialbehaviour} on page \pageref{subsec:socialbehaviour} for details) they had the idea to design an automated system, to collect spatial position data of mice living in a shed.

\subsection{Motivation}
\label{subsec:motivation}
The motivation for such an effort is to obtain a statistically relevant set of data to develop, test and prove scientific thesis about the complex social behaviour of the house mice. The research interests are outlined in section \ref{subsec:researchtopics} on page \pageref{subsec:researchtopics}. An automated system has the advantage that it collects data 24 hours a day, and the mice are in no way disturbed in their natural behavior by the presence of men.

My personal motivation comes from the interest in informatics and my study in biology. The project gave me a lot of room to enhance my skills in informatics, while working on an interesting biological topic. Additionally I am interested in the \textit{network approach} to analyze complex social behaviour by the use of \textit{Graph theory}\cite{wiki:graph_theory}, which is outlined in section \ref{sec:netexp} on page \pageref{sec:netexp}.

The system collecting the data in the shed has been developed and constructed by \textit{New Behaviour}. For further information about the design and operation of the system please see section \ref{subsec:collectspatialpos} on page \pageref{subsec:collectspatialpos}. Thus, the main task for my master thesis was to provide tools to store the data and make it accessible to the users, which are mainly the researchers involved in the house mice project. 

\subsection{Task}
\label{subsec:task}
Handling the spatial position data collected by the automated system is challenging, as it includes several tasks which can be carried out only using informatics on a level that biologists normally can not cope with.

\subsubsection{Storing and Accessing the Data}
\label{subsubsec:storeaccess}
The sheer amount of data, expected to be collected by the automated system, has to be stored in a secure and stable way. The only reasonable way to fulfill these conditions, is to set up a \ac{RDBMS} (RDBMS) (see section \ref{subsec:datastorage} on page \pageref{subsec:datastorage}). 

To access the data, an intuitive user interface must be available, which facilitates the data exploration and export (see section \ref{subsec:dataccess} on page \pageref{subsec:dataccess}). The technology used to create the interface should be cross-platform compatible.   

\subsubsection{Additional Tasks}
\label{subsubsec:additional}
Additionally, the user should be provided with an interface to administrate attribute data, such as the sex of the mice (see section \ref{subsec:dataattr} on page \pageref{subsec:dataattr}) and to carry out some simple data analysis methods.

Last but not least, some functionality should be present which allows the user to compile data for a possible analysis according to the network approach (see section \ref{sec:netexp} on page \pageref{sec:netexp}).